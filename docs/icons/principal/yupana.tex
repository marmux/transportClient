%% icon for yupana gps tracker
%% author Marco Muñiz Jul-2016

\documentclass{article}

\usepackage{tikz}   
\usepackage[active,tightpage]{preview}
\usepackage{pgfplots}
\usepackage{xstring}

\usetikzlibrary{arrows,shapes}
\usetikzlibrary{automata}
\usetikzlibrary{positioning,shadows}
\usetikzlibrary{decorations.pathmorphing,snakes}
\usetikzlibrary{decorations.markings}
\usetikzlibrary{fit}					% fitting shapes to
\usetikzlibrary{matrix}
\usetikzlibrary{backgrounds}


\setlength\PreviewBorder{1pt}
\PreviewEnvironment{tikzpicture}
\begin{document}

\setlength{\unitlength}{1cm}
\def\numRows{4}
\def\numCols{4}
\def\laneWidth{2}
\def\radiusSignal{0.15}
\def\maisColor{blue!90}
\def\frameColor{blue!100}
\newcount\xori
\newcount\yori
\newcount\xdes
\newcount\ydes
\newcount\xpos
\newcount\ypos
\newcount\xincrement
\newcount\yincrement
\newcount\xcenter
\newcount\ycenter


\def\trafficLightOffSetx{\laneWidth*0}
\def\trafficLightOffSety{\laneWidth*0}

\def\xorigin{0}
\def\yorigin{0}


\tikzset{
  yupanaFrame/.style={very thick,color=\frameColor,step=\laneWidth},
  maisFull/.style={\maisColor,fill,thick},
  maisEmpty/.style={\maisColor,very thick},
  empty/.style={rectangle,thick,draw=none,fill=none},
}


%%%%%%%%%%%%%%%%%%%%% COMMANDS %%%%%%%%%%%%%%%%%%%%%%%%%%%

\newcommand{\drawMais}[3]{%grid i,j and full or empty
 \IfStrEqCase{#3}{%
    {FULL}{%
      \draw[maisFull] (#1,#2) circle (\radiusSignal);
    }
    {EMPTY}{%
      \draw[maisEmpty] (#1,#2) circle (\radiusSignal);
    }
  }
}

\newcommand{\drawLogo}[2]{%yupana labs logo
  %we compute the top left coorner of a square from the origin
  \pgfmathparse{(#1*\laneWidth)}  \let\xori\pgfmathresult
  \pgfmathparse{(#2*\laneWidth)}  \let\yori\pgfmathresult
  % we compute the center of the square
  \pgfmathparse{(\xori+\laneWidth-\laneWidth/2)}  \let\xpos\pgfmathresult
  \pgfmathparse{(\yori-\laneWidth+\laneWidth/8)}  \let\ypos\pgfmathresult
  \node[empty] at (\xpos,\ypos) {\small \color{\maisColor}{YupanaLabs}};
}

\newcommand{\drawUno}[3]{%grid i, j in x#1 and y#2
  %we compute the top left coorner of a square from the origin
  \pgfmathparse{(#1*\laneWidth)}  \let\xori\pgfmathresult
  \pgfmathparse{(#2*\laneWidth)}  \let\yori\pgfmathresult
  % we compute the center of the square
  \pgfmathparse{(\xori+\laneWidth/2)}  \let\xcenter\pgfmathresult
  \pgfmathparse{(\yori-\laneWidth/2)}  \let\ycenter\pgfmathresult
  \drawMais{\xcenter}{\ycenter}{#3}
}


\newcommand{\drawDos}[4]{%grid i, j in x#1 and y#2
  %we compute the top left coorner of a square from the origin
  \pgfmathparse{(#1*\laneWidth)}  \let\xori\pgfmathresult
  \pgfmathparse{(#2*\laneWidth)}  \let\yori\pgfmathresult
  % we compute the x center of the square
  \pgfmathparse{(\xori+\laneWidth/2)}  \let\xcenter\pgfmathresult
  % we compute the y possitions
  \pgfmathparse{(\laneWidth)/6)}  \let\yincrement\pgfmathresult
  %mais 1
  \pgfmathparse{\yori-(\yincrement*2)-\trafficLightOffSety} \let\ypos\pgfmathresult
  \drawMais{\xcenter}{\ypos}{#3}
  \pgfmathparse{\yori-(\yincrement*4)-\trafficLightOffSety} \let\ypos\pgfmathresult
  \drawMais{\xcenter}{\ypos}{#4}
}

\newcommand{\drawTres}[5]{%grid i, j in x#1 and y#2
  %we compute the top left coorner of a square from the origin
  \pgfmathparse{(#1*\laneWidth)} \let\xori\pgfmathresult
  \pgfmathparse{(#2*\laneWidth)} \let\yori\pgfmathresult
  % we compute the x, y increment
  \pgfmathparse{(\laneWidth)/6)} \let\xincrement\pgfmathresult
  \pgfmathparse{(\laneWidth)/6)} \let\yincrement\pgfmathresult
  % mais 1 2
  \pgfmathparse{\xori+(\xincrement*2)-\trafficLightOffSetx} \let\xpos\pgfmathresult
  \pgfmathparse{\yori-(\yincrement*2)-\trafficLightOffSety} \let\ypos\pgfmathresult
  \drawMais{\xpos}{\ypos}{#3}  
  \pgfmathparse{\yori-(\yincrement*4)-\trafficLightOffSety} \let\ypos\pgfmathresult
  \drawMais{\xpos}{\ypos}{#4}
  % mais 3
  \pgfmathparse{\xori+(\xincrement*4)-\trafficLightOffSetx} \let\xpos\pgfmathresult
  \pgfmathparse{(\yori-\laneWidth/2)} \let\ycenter\pgfmathresult
  \drawMais{\xpos}{\ycenter}{#5}  
}

\newcommand{\drawCinco}[7]{%grid i, j in x#1 and y#2
  %we compute the top left coorner of a square from the origin
  \pgfmathparse{(#1*\laneWidth)} \let\xori\pgfmathresult
  \pgfmathparse{(#2*\laneWidth)} \let\yori\pgfmathresult
  % we compute the x, y increment
  \pgfmathparse{(\laneWidth)/6)} \let\xincrement\pgfmathresult
  \pgfmathparse{(\laneWidth)/8)} \let\yincrement\pgfmathresult
  % mais 1 2 3
  \pgfmathparse{\xori+(\xincrement*2)-\trafficLightOffSetx} \let\xpos\pgfmathresult
  \pgfmathparse{\yori-(\yincrement*2)-\trafficLightOffSety} \let\ypos\pgfmathresult
  \drawMais{\xpos}{\ypos}{#3}  
  \pgfmathparse{\yori-(\yincrement*4)-\trafficLightOffSety} \let\ypos\pgfmathresult
  \drawMais{\xpos}{\ypos}{#4}
  \pgfmathparse{\yori-(\yincrement*6)-\trafficLightOffSety} \let\ypos\pgfmathresult
  \drawMais{\xpos}{\ypos}{#5}
  % mais 4 5
  \pgfmathparse{\xori+(\xincrement*4)-\trafficLightOffSetx} \let\xpos\pgfmathresult
  \pgfmathparse{\yori-(\yincrement*3)-\trafficLightOffSety} \let\ypos\pgfmathresult
  \drawMais{\xpos}{\ypos}{#6}
  \pgfmathparse{\yori-(\yincrement*5)-\trafficLightOffSety} \let\ypos\pgfmathresult
  \drawMais{\xpos}{\ypos}{#7}  
}




\begin{tikzpicture}%[xscale=1,yscale=0.8]
   \pgfmathparse{((\numCols)*\laneWidth)}
   \let\xdes\pgfmathresult
   \pgfmathparse{((\numRows)*\laneWidth)}
   \let\ydes\pgfmathresult
   \draw[yupanaFrame] (0,0) grid (\xdes,\ydes);

   \drawUno{3}{1}{FULL}
   \drawUno{3}{2}{EMPTY}
   \drawUno{3}{3}{FULL}
   \drawUno{3}{4}{EMPTY}
   \drawDos{2}{1}{FULL}{EMPTY}
   \drawDos{2}{2}{FULL}{EMPTY}
   \drawDos{2}{3}{EMPTY}{FULL}
   \drawDos{2}{4}{EMPTY}{EMPTY}
   \drawTres{1}{1}{EMPTY}{EMPTY}{EMPTY}
   \drawTres{1}{2}{FULL}{EMPTY}{EMPTY}
   \drawTres{1}{3}{EMPTY}{FULL}{FULL}
   \drawTres{1}{4}{EMPTY}{FULL}{FULL}
   \drawCinco{0}{1}{EMPTY}{EMPTY}{EMPTY}{FULL}{EMPTY}
   \drawCinco{0}{2}{FULL}{EMPTY}{EMPTY}{FULL}{FULL}
   \drawCinco{0}{3}{EMPTY}{FULL}{FULL}{FULL}{EMPTY}
   \drawCinco{0}{4}{EMPTY}{FULL}{FULL}{EMPTY}{FULL}
   \drawLogo{3}{1}
   
\end{tikzpicture}    


\end{document}

